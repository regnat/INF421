%%%%%%%%%%%%%%%%%%%%%%%%%%%%%%%%%%%%%%%%%%%%%%%%%%
% presentation.tex
%
% Présentation du proet d'info
%
%%%%%%%%%%%%%%%%%%%%%%%%%%%%%%%%%%%%%%%%%%%%%%%%%%

\documentclass{beamer}
\usetheme{Hannover}
\usecolortheme{seahorse}
\usefonttheme{structurebold}
\usepackage[utf8]{inputenc}
\beamertemplatenavigationsymbolsempty{}

\usepackage[frenchb]{babel}
\usepackage{graphicx}
\usepackage{mathtools}
\usepackage[]{algorithm2e}
\usepackage{autoTitle}

\title{À la recherche de clés secrètes vulnérables}
\subtitle{PI02}
\date{}
\author{Théophane Hufschmitt\\Guillaume Hétier}
\begin{document}
\beamerdefaultoverlayspecification{<+->}
\setbeamercovered{dynamic}

\section{Arbre des produits et des restes}
\subsection{Arbre des produits}
\begin{frame}
  \includegraphics{img/arbreProd}  
\end{frame}

\subsection{Complexité de construction}
\begin{frame}
  \begin{block}{Complexité de la construction de l'arbre des produits}
  On numérote les étages de l'arbre de $0$ pour les feuilles à $k$ pour la racine. 
   \begin{itemize}
    \item Étage interne de l'arbre~: une multiplication par nœuds, soit $2^{k-i}$ multiplications.
    \item Nombre de chiffres d'un nombre au niveau $k$~: double au plus à chaque niveau, soit $2^{i-1}m$ chiffres.
    \item Au total~: $C(2^k,m) = \sum_{i=1}^k 2^{k-i}CM(2^{i-1}m)$
   \end{itemize}
  \end{block}
  \begin{block}{Complexité de la construction de l'arbre des restes}
    \begin{itemize}
    \item Étage de l'arbre autre que le sommet~: une division par nœuds, soit $2^{k-i}$ divisions.
    \item Nombre de chiffres d'un nombre au niveau $k$~: division par un nombre présent à l'étage d'en dessous dans l'arbre des produits, qui a donc $2^{i}m$ chiffres au plus. Donc le reste a $2^{i}m$ chiffres au plus.
    \item Au total~: $C(2^k,m) = \sum_{i=0}^{k-1} 2^{k-i}CD(2^{i+1}m)$
   \end{itemize}
  \end{block}
\end{frame}

\begin{frame}
 \begin{block}{Comparaison des complexité selon la complexité des opérations}
  Pour des opérations en temps~:
  \begin{itemize}
   \item quadratique~: $C(N,m) = O(N^2m^2)$
   \item amélioré~: $C(N) = O(Nm\log(N)\log(Nm)\log(\log(Nm)))$
  \end{itemize}
 \end{block}
 
 On gagne quasiment un ordre de grandeur.

\section{Obtention des facteurs communs}
\end{frame}
  
\begin{frame}
   À partir des feuilles de l'arbre des restes, on calcule $x = pgcd(r_i/p_i,pi)$.
   \begin{block}
   \begin{itemize}
    \item Si $x=1$~: la clé ne partage aucun de ses facteurs.
    \item Si $x \leq 1$~: la clé a un unique facteur commun, qui vaut $x$.
    \item Si $x=0$~: la clé a ses deux facteurs en commun, mais on ne peux pas avoir d'informations dessus.
   \end{itemize}
   \end{block}

   
\end{frame}


\end{document}
