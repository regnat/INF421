%%%%%%%%%%%%%%%%%%%%%%%%%%%%%%%%%%%%%%%%%%%%%%%%%%
% presentation.tex
%
% Présentation du proet d'info
%
%%%%%%%%%%%%%%%%%%%%%%%%%%%%%%%%%%%%%%%%%%%%%%%%%%

\documentclass{beamer}
\usetheme{Hannover}
\usecolortheme{seahorse}
\usefonttheme{structurebold}
\usepackage[utf8]{inputenc}
\beamertemplatenavigationsymbolsempty{}

\usepackage[frenchb]{babel}
\usepackage{graphicx}
\usepackage{mathtools}
\usepackage[]{algorithm2e}
\usepackage{autoTitle}

\title{À la recherche de clés secrètes vulnérables}
\subtitle{PI02}
\date{}
\author{Théophane Hufschmitt\\Guillaume Hétier}
\begin{document}
\beamerdefaultoverlayspecification{<+->}
\setbeamercovered{dynamic}
\begin{frame}
  \maketitle
\end{frame}

\section{Calcul naïf de pgcd}
\subsection{Justification théorique}
\begin{frame}
  \begin{block}{Principe des clés RSA}
    \begin{itemize}
      \item Clé privée~: Paire de grands nombres premiers $p$ et $q$
      \item Clé publique~: Produit $n$ de $p$ et $q$
      \item Factorisation de la clé publique très difficile
    \end{itemize}
  \end{block}
  \begin{block}{Faiblesse}
    \begin{itemize}
      \item Très grande quantité de clés existantes
      \item Risque de ``collision'' sur une des clés privées
    \end{itemize}
  \end{block}
\end{frame}
\subsection{Réalisation pratique}
\begin{frame}
  \begin{block}{Principe de l'algorithme}
    \begin{itemize}
      \item Simple itération sur chaque paire de clés en entrée
      \item Requiert $\mathcal{O}(n^2)$ calculs de pgcd
    \end{itemize}
  \end{block}  
  \begin{block}{Pseudo-code}
    \begin{algorithm}[H]
      \Entree{keys: Liste de clés publiques}
      \Pour{key1, key2 $\in$ keys}{%
        \Si{gcd (key1, key2)~!= 1 \&\& key1~!= key2}{%
          ajouter (key1, key2, gcd) \tcc*{key1 et key2 sont factorisées}
        }
      }
    \end{algorithm}
  \end{block}
\end{frame}
\end{document}
